\documentclass{indiabill}
\billtitle{Public Health Emergency Bill}
\billday{5}
\billmonth{March}
\billyear{2024} 
\billto{to provide a regulatory framework for the declaration and management of public health emergencies in a manner that protects individual liberties and secures the health of persons}
\vspace{1ex}

\billnum{00/01}

\begin{document}

\section{Title, extent, and commencement}
	\begin{numstat}
		\item This Act will be called the Public Health Emergencies Act, 2024.
		\item This Act extends to the whole of India.
		\item This Act will come into force from the date notified by the Central Government.
	\end{numstat}

\section{Definitions}
	\begin{numstat}
	\item "Act" means the Public Health Emergencies Act 2024 and all rules, regulations and bye-laws made under it.
	\end{numstat}

\section{Establishment of Public Health Agencies}
	\begin{numstat}
	\item The following Public Health Agencies are established to exercise the powers and
	discharge the functions assigned to them under this Act, -- 
		\begin{alphstat}
		\item Disease Control and Prevention Authority
		\item Epidemic Intelligence Center 
		\end{alphstat}
	\item Each Public Health Agency will be a body corporate having, --
		\begin{alphstat}
		\item perpetual succession; and 
		\item a common seal. 
		\end{alphstat}
	\item Subject to the provisions of this Act, each Public Health Agency will have the power and ability to, --
		\begin{alphstat}
		\item enter and execute contracts;
		\item acquire, hold and dispose of property, both movable and immovable; and
		\item sue and be sued.
		\end{alphstat}
	\item The Disease Control and Prevention Authority will have its head office in Mumbai. 
	\item The Epidemic Intelligence Center will have its head office in Bengaluru and will have offices in every district in India. 
	\end{numstat}

\section{Functions of Public Health Agencies}
	\begin{numstat}
	\item The Disease Control and Prevention Authority will coordinate and regulate the prevention of and response to a public health emergency. 
	\item The Epidemic Intelligence Center will provide the service of epidemiological monitoring and measurement throughout India.
	\end{numstat}

\section{Overriding the Disaster Management Act}
	\begin{numstat}
	\item The Disaster Management Act 2005 shall not apply in the event of a public health emergency.
	\end{numstat}

\section{Repeal and Savings}
	\begin{numstat}
	\item The Epidemic Diseases Act 1897 is repealed.
	\end{numstat}

\end{document}
